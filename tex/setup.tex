\usepackage{amsfonts}
\usepackage{amsmath,amsthm,amssymb,graphicx,setspace,url,booktabs,tabularx,enumerate,slantsc}
\usepackage[T1]{fontenc}
\usepackage[nolists,nomarkers]{endfloat}
\usepackage[sort,round,comma]{natbib}
\usepackage[margin=1.25in]{geometry}
\usepackage[small]{caption}
\usepackage[charter]{mathdesign}
\urlstyle{same}
\newcolumntype{C}{>{\centering\arraybackslash}X}
\bibliographystyle{abbrvnat}
\newcommand\citepos[2][]{\citeauthor{#2}'s \citeyearpar[#1]{#2}}
\newcommand\poscw{\citeauthor{ClW:06}'s \citeyearpar{ClW:06,ClW:07}}
\newcommand\citen[1]{\citeauthor{#1}, \citeyear{#1}}
\frenchspacing
\newcommand{\empiricalcriticalvalue}{2.67}
\newcommand{\spacriticalvalue}{1.26}
\newcommand{\nboot}{599}
\newcommand{\bootsize}{10}
\newcommand{\windowlength}{10}
\newcommand{\empiricaltable}{% latex table generated in R 3.2.0 by xtable 1.7-4 package
% Mon Apr 20 13:17:43 2015
\begin{tabularx}{\textwidth}{lCCCC}
  \toprule   & value & naive & SPA & ours \\ 
  \midrule long term rate & $\enskip1.56$ & sig. & sig. &  \\ 
  book to market & $\enskip1.41$ & sig. & sig. &  \\ 
  dividend yield & $\enskip1.27$ &  & sig. &  \\ 
  dividend price ratio & $\enskip0.95$ &  &  &  \\ 
  net equity & $\enskip0.70$ &  &  &  \\ 
  dividend payout ratio & $\enskip0.64$ &  &  &  \\ 
  treasury bill & $\enskip0.53$ &  &  &  \\ 
  stock variance & $\enskip0.50$ &  &  &  \\ 
  default return spread & $\enskip0.16$ &  &  &  \\ 
  default yield spread & $\enskip0.09$ &  &  &  \\ 
  inflation & $\!\!-0.09$ &  &  &  \\ 
  term spread & $\!\!-0.43$ &  &  &  \\ 
  earnings price ratio & $\!\!-0.56$ &  &  &  \\ 
  long term yield & $\!\!-0.74$ &  &  &  \\ 
   \bottomrule \end{tabularx}
}


% These commands are generated when the monte carlo and applied
% sections are run; I'm giving them definitions now so that LaTeX will
% run.
\providecommand\bootsize{[missing]}
\providecommand\empiricalcriticalvalue{[missing]}
\providecommand\nboot{[missing]}
\providecommand\testsize{[missing]}
\providecommand\totalsims{[missing]}
\providecommand\windowlength{[missing]}
\providecommand\empiricaltable{[missing]}

\newtheorem{thm}{Theorem}
\newtheorem{lem}[thm]{Lemma}
\newtheorem{claim}[thm]{Claim}
\newtheorem{cor}[thm]{Corollary}
\newtheorem{lema}{Lemma}[subsection]
\newtheorem{alg}{Algorithm}
\newtheorem{asmp}{Assumption}
\theoremstyle{definition}

\newtheorem{example}{Example}
\newtheorem{defn}{Definition}
\newtheorem{rem}{Remark}

\DeclareMathOperator*{\argmin}{arg\,min}
\DeclareMathOperator{\E}{E}
\DeclareMathOperator{\var}{var}
\DeclareMathOperator{\cov}{cov}
%\DeclareMathOperator{\vec}{vec}
\DeclareMathOperator{\vech}{vech}

\DeclareMathOperator{\pr}{Pr}

\newcommand{\btrue}{\beta_0}

\newcommand{\X}{\ensuremath{\mathrm{X}}}
\newcommand{\R}{\ensuremath{\mathrm{R}}}
\newcommand{\p}{\ensuremath{\mathrm{P}}}

\newcommand{\bh}{\hat \beta}
\newcommand{\ep}{\varepsilon}
\newcommand{\eph}{\hat\varepsilon}
\newcommand{\fb}{\bar f}
\newcommand{\fh}{\hat f}
\newcommand{\Hs}{\mathcal{H}}

\newcommand{\osum}[1]{\sum_{#1=R}^{T-1}}
\newcommand{\omax}[1]{\max_{#1=R,\dots,T-1}}
\newcommand{\oavg}[1]{\tfrac{1}{P} \osum{#1}}
\newcommand{\oclt}[1]{\tfrac{1}{\sqrt{P}} \osum{#1}}

\newcommand{\aic}{AIC}
\newcommand{\bic}{BIC}
\newcommand{\brc}{BRC}
\newcommand{\cdf}{CDF}
\newcommand{\clt}{CLT}
\newcommand{\dd}[1]{\frac{\partial}{\partial #1}}
\newcommand{\dgp}{DGP}
\newcommand{\fclt}{FCLT}
\newcommand{\fwe}{FWE}
\newcommand{\gdp}{GDP}
\newcommand{\gmm}{GMM}
\newcommand{\hac}{HAC}
\newcommand{\iv}{IV}
\newcommand{\lln}{LLN}
\newcommand{\ma}{MA}
\newcommand{\mds}{MDS}
\newcommand{\ned}{NED}
\newcommand{\ols}{OLS}
\newcommand{\oos}{OOS}
\newcommand{\sfwe}{SFWE}
\newcommand{\spa}{SPA}
\newcommand{\wfwe}{WFWE}

\renewcommand{\Re}{\ensuremath{\mathbb{R}}}

\renewcommand{\topfraction}{.85}
\renewcommand{\bottomfraction}{.7}
\renewcommand{\textfraction}{.15}
\renewcommand{\floatpagefraction}{.66}
\renewcommand{\dbltopfraction}{.66}
\renewcommand{\dblfloatpagefraction}{.66}
\setcounter{topnumber}{9}
\setcounter{bottomnumber}{9}
\setcounter{totalnumber}{20}
\setcounter{dbltopnumber}{9}
